\documentclass{article}
\usepackage[utf8]{inputenc}
 
\usepackage[
backend=biber,
style=alphabetic,
sorting=ynt
]{biblatex}
\addbibresource{grep-parabix.bib}

\title{Reinventing Grep with Parabix Technology}
\begin{document}
\maketitle

\section{Introduction}

Searching through text files for strings matching a particular
pattern is a common task in information processing.   
The software tool grep was developed by Ken Thompson for this
purpose and made available with Unix Version 4 in 1973.
\cite{mcilroy1987research}.
The name grep arises from the g/re/p command of Thomposon's
ed editor meaning globally search for regular expression and print.
While Thompson's initial version did not support the regular
expression features of alternation and grouping, this was
remedied by the development of egrep by Alfred Aho in 1975 \cite{hume1988tale}.
Andrew Hume improved egrep performance
by incorporating the Boyer-Moore algorithm for fixed strings in 1988.
Free and open source software versions of grep were developed
by the GNU and BSD projects.   Bitwise parallel NFA simulation 
was introduced with agrep \cite{wu1992agrep} and nrgrep \cite{navarro2001nr}.   

All of these grep implementations are fundamentally sequential in
nature, primarily building on automata theory techniques involving
deterministic or nondeterministic finite automata (NFAs or DFAs).
In general, the searches proceed byte-at-a-time updating the 
automaton state (or set of states in the case of NFAs) with each step.
The byte-at-a-time processing model was well matched to processor
capabilities at the time of initial grep development, but 
processors have evolved to be able to process many bytes in parallel 
using short vector SIMD instructions.   For more than two decades,
processing 16 bytes at a time has been possible with the Intel 
SSE instructions, Power PC Altivec instructions or
ARM Neon instructions.   In the last several years, Intel has further
increased SIMD register width, first with 32-byte AVX2 technology 
and most recently with 64-byte AVX-512 technology.   

Parabix technology is a programming framework under development by
our research group at Simon Fraser University  to take
advantage of the wide SIMD registers for streaming text processing
applications such as grep \cite{lin2012parabix}.   Rather than focussing on byte-oriented
processing, however, it uses the concept of bitwise data parallelism.
In this model, bit positions within SIMD registers are associated
with byte positions in input data streams.   With AVX2 technology,
for example, the goal is to process 256 bytes at a time with the
256-bit SIMD registers.   We have applied Parabix technology 
to accelerate Unicode transcoding \cite{cameron2008case}, XML parsing \cite{cameron2011parallel, medforth2013icxml} and regular
expression search  \cite{cameron2014bitwise}, while other groups have applied related techniques
to accelerate RNA protein search  \cite{peace2010exact} and JSON parsing \cite{li2017mison} .

As a showcase of the Parabix framework, icgrep is a new grep
implementation fundamentally built using bitwise data parallel
techniques.   With our promising performance results in icgrep 1.0 \cite{cameron2015bitwise},
we have focused on building a fully modern grep with broad
support of extended regular expression features including 
the lookaround assertions of Perl-compatible regular expressions
and the full set of Unicode level 2 regular expression features
defined by the Unicode consortium \cite{davis2016unicode}.   We believe icgrep is the first
grep implementation to achieve this level of Unicode support.
In addition, we have also focused on incorporating educational
features so that users can explore and display various transformations
that take placed during regular expression processing and/or 
Parabix compilation.

From a performance perspective, we have considered three important
aspects for modern software tools.    One aspect is the
use of fundamentally parallel algorithms as well as the systematic
use of SIMD and multicore parallelism.   The second aspect is to
focus on scalability, that is to arrange to automatically take
advantage of both available SIMD register width and available
cores to achieve performance that scales with these resources.
While there is still considerable work to do in the development of 
our framework, icgrep does indeed demonstrate such scalability, 
even for single file search.
From this perspective, we think that icgrep represents an interesting
initial data point with respect to software tool scalability.
Finally, we also have focused on consistent and predictable performance,
particularly in response to the rise in denial-of-service attacks
that exploit pathological cases for many existing regular expression
tools \cite{kirrage2013static}.

While research continues, our goal is to introduce icgrep as
an important practical contribution that pushes the boundaries
on both the performance and capability of grep search software.
As with all open source software projects, we expect to refine
the software over time and invite collaborators to join our
effort.

\section{Parabix Regular Expression Matching}


\printbibliography

\end{document}




